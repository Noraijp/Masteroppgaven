%%%%%%%%%%%%%%%%%%%%%%%%%%%%%%%%%%%%%%%%%%%%%%
%%%%%%%%%%%%%%%%%%%%%%%%%%%%%%%%%%%%%%%%%%%%%%
%%                                          %%
%% Important note on usage                  %%
%% -----------------------                  %%
%% This file must be compiled with PDFLaTeX %%
%% Using standard LaTeX will not work!      %%
%%                                          %%
%%%%%%%%%%%%%%%%%%%%%%%%%%%%%%%%%%%%%%%%%%%%%%
%%%%%%%%%%%%%%%%%%%%%%%%%%%%%%%%%%%%%%%%%%%%%%

%% The '3p' and 'times' class options of elsarticle are used for Elsevier CRC
\documentclass[3p]{elsarticle}
% \documentclass[5p]{elsarticle}

\usepackage[american]{babel}
\usepackage{amsmath}
\usepackage[version=3]{mhchem} 
% \usepackage{fixltx2e}
% \usepackage{refcount}
% \usepackage{siunitx}
% \usepackage{lastpage}
% \usepackage{textcomp}
\usepackage{mathtools}

\usepackage{xfrac}
\usepackage{lmodern}
\usepackage[hidelinks]{hyperref}
% \usepackage{cool}
% \usepackage{cancel}
\usepackage{microtype}
\usepackage{listings}
%\usepackage{mcode}
\usepackage[autostyle, english = american]{csquotes}
\usepackage{longtable}
% \usepackage{subcaption}
\usepackage{booktabs,siunitx}
\usepackage{gensymb}
\usepackage[normalem]{ulem}

% \usepackage{mathtools, cuted}
\usepackage{lipsum} % for dummy text only


% \usepackage[usenames,dvipsnames,svgnames,table]{xcolor}
\usepackage{color}

\usepackage[colorinlistoftodos]{todonotes}

\usepackage[section]{placeins}
% \usepackage{multirows}


\lstset{basicstyle=\small\ttfamily,columns=fullflexible}

% \usepackage{verbatim}



% \usepackage{gensymb}
% \usepackage{enumerate}
% \usepackage{float}
% \usepackage{bm}
% \usepackage{csquotes}
% \usepackage{mathtools}
% \usepackage{natbib}
% \usepackage{biblatex}

%% The `ecrc' package must be called to make the CRC functionality available
\usepackage{ecrc-ari}

%% The ecrc package defines commands needed for running heads and logos.
%% For running heads, you can set the journal name, the volume, the starting page and the authors

%% set the volume if you know. Otherwise `00'
\volume{00}

%% set the starting page if not 1
\firstpage{1}

%% Give the name of the journal
\journalname{Applied Radiation and Isotopes}

%% Give the author list to appear in the running head
%% Example \runauth{C.V. Radhakrishnan et al.}
\runauth{Your Name et al.}

%% The choice of journal logo is determined by the \jid and \jnltitlelogo commands.
%% A user-supplied logo with the name <\jid>logo.pdf will be inserted if present.
%% e.g. if \jid{yspmi} the system will look for a file yspmilogo.pdf
%% Otherwise the content of \jnltitlelogo will be set between horizontal lines as a default logo

%% Give the abbreviation of the Journal.
\jid{ari}
% \jid{yspmi}

%% Give a short journal name for the dummy logo (if needed)
% \jnltitlelogo{Nucl Instrum Meth B}

%% Hereafter the template follows `elsarticle'.
%% For more details see the existing template files elsarticle-template-harv.tex and elsarticle-template-num.tex.

%% Elsevier CRC generally uses a numbered reference style
%% For this, the conventions of elsarticle-template-num.tex should be followed (included below)
%% If using BibTeX, use the style file elsarticle-num.bst

%% End of ecrc-specific commands
%%%%%%%%%%%%%%%%%%%%%%%%%%%%%%%%%%%%%%%%%%%%%%%%%%%%%%%%%%%%%%%%%%%%%%%%%%

%% The amssymb package provides various useful mathematical symbols
\usepackage{amssymb}
%% The amsthm package provides extended theorem environments
\usepackage{amsthm}
%\usepackage{amsrefs}

%% The lineno packages adds line numbers. Start line numbering with
%% \begin{linenumbers}, end it with \end{linenumbers}. Or switch it on
%% for the whole article with \linenumbers after \end{frontmatter}.
%% \usepackage{lineno}

%% natbib.sty is loaded by default. However, natbib options can be
%% provided with \biboptions{...} command. Following options are
%% valid:

%%   round  -  round parentheses are used (default)
%%   square -  square brackets are used   [option]
%%   curly  -  curly braces are used      {option}
%%   angle  -  angle brackets are used    <option>
%%   semicolon  -  multiple citations separated by semi-colon
%%   colon  - same as semicolon, an earlier confusion
%%   comma  -  separated by comma
%%   numbers-  selects numerical citations
%%   super  -  numerical citations as superscripts
%%   sort   -  sorts multiple citations according to order in ref. list
%%   sort&compress   -  like sort, but also compresses numerical citations
%%   compress - compresses without sorting
%%
%% \biboptions{comma,round}

\biboptions{sort&compress}

% if you have landscape tables
\usepackage[figuresright]{rotating}

% put your own definitions here:
%   \newcommand{\cZ}{\cal{Z}}
%   \newtheorem{def}{Definition}[section]
%   ...

% add words to TeX's hyphenation exception list
%\hyphenation{author another created financial paper re-commend-ed Post-Script}

% declarations for front matter

\usepackage{fancyvrb}
\usepackage{color}
 
\definecolor{mygreen}{rgb}{0,0.6,0}
\definecolor{mygray}{rgb}{0.5,0.5,0.5}
\definecolor{mymauve}{rgb}{0.58,0,0.82}

\lstset{ %
  backgroundcolor=\color{white},   % choose the background color
  basicstyle=\footnotesize,        % size of fonts used for the code
  breaklines=true,                 % automatic line breaking only at whitespace
  captionpos=b,                    % sets the caption-position to bottom
  commentstyle=\color{mygreen},    % comment style
  escapeinside={\%*}{*)},          % if you want to add LaTeX within your code
  keywordstyle=\color{blue},       % keyword style
  stringstyle=\color{mymauve},     % string literal style
}

% Sin and Cos with auto-parentheses 
\newcommand{\sinp}[1]{\sin{\left( #1\right)}}
\newcommand{\cosp}[1]{\cos{\left( #1\right)}}
\newcommand{\expp}[1]{\exp{\left( #1\right)}}
\newcommand{\sinhp}[1]{\sinh{\left( #1\right)}}
\newcommand{\lnp}[1]{\ln{\left( #1\right)}}
\newcommand{\pp}[1]{\left( #1\right)}
\newcommand{\sci}[2]{ #1 \cdot 10^{#2}\ }
\newcommand{\angstrom}{\mbox{\normalfont\AA}}
\newcommand{\norm}[1]{\lVert #1 \rVert}

\newcommand{\textred}[1]{\textcolor{red}{ #1}}
\newcommand{\redactedit}[1]{\textcolor{blue}{ \sout{#1}}}


\newcommand{\colornote}[1]{\textcolor{red}{ COMMENT\large\footnote{\textcolor{red}{#1}}}}

\newcommand{\comment}[1]{\todo[color=blue!20!white,inline]{ASV: #1}} 

% Tweak sim for better inline text tilde
\newcommand{\mytilde}{\raisebox{0.5ex}{\texttildelow}}
% \newcommand{\mytilde}{\raise.17ex\hbox{$\scriptstyle‌​\sim$}}

% \sisetup{separate-uncertainty=true,table-space-text-post = *}

\newcommand{\minitab}[2][l]{\begin{tabular}{#1}#2\end{tabular}}


\newcommand{\subfigimg}[4][,]{%
  \setbox1=\hbox{\includegraphics[#1]{#3}}% Store image in box
  \leavevmode\rlap{\usebox1}% Print image
  \rlap{\hspace*{#4pt}\raisebox{\dimexpr\ht1-2\baselineskip}{#2}}% Print label
  \phantom{\usebox1}% Insert appropriate spcing
}
% Old version of macro
% \newcommand{\subfigimg}[3][,]{%
%   \setbox1=\hbox{\includegraphics[#1]{#3}}% Store image in box
%   \leavevmode\rlap{\usebox1}% Print image
%   \rlap{\hspace*{50pt}\raisebox{\dimexpr\ht1-2\baselineskip}{#2}}% Print label
%   \phantom{\usebox1}% Insert appropriate spcing
% }
\usepackage{subfig}
% Remove a), b), etc labels from subfigs
\captionsetup[subfigure]{labelformat=empty}



\makeatletter
% Make common definition of mean
\newcommand*\mean[1]{\overline{#1\raisebox{3mm}{}}}

\makeatother


\begin{document}

\begin{frontmatter}

%% Title, authors and addresses

%% use the tnoteref command within \title for footnotes;
%% use the tnotetext command for the associated footnote;
%% use the fnref command within \author or \address for footnotes;
%% use the fntext command for the associated footnote;
%% use the corref command within \author for corresponding author footnotes;
%% use the cortext command for the associated footnote;
%% use the ead command for the email address,
%% and the form \ead[url] for the home page:
%%
\title{Your Title Here}


% \dochead{Short}
%% Use \dochead if there is an article header, e.g. \dochead{Short communication}



\author[uio]{Your Name}

\author[ucb]{Andrew S. Voyles \corref{cor1}}
\ead{andrew.voyles@berkeley.edu}


\author[ucb,lbl]{Lee A. Bernstein}

\author[uio]{Sunniva Siem}







%% use optional labels to link authors explicitly to addresses:
%% \author[label1,label2]{<author name>}
%% \address[label1]{<address>}
%% \address[label2]{<address>}

\cortext[cor1]{Corresponding author}

% \address[ucb]{Department of Nuclear Engineering, University of California, Berkeley, Etcheverry Hall, 2521 Hearst Ave, Berkeley, CA 94709}
% \address[lbl]{Lawrence Berkeley National Laboratory,  1 Cyclotron Rd, Berkeley, CA 94720}
% \address[llnl]{Lawrence Livermore National Laboratory, 7000 East Ave, Livermore, CA 94550}

\address[uio]{Department of Physics, University of Oslo, N-0316 Oslo, Norway}
\address[ucb]{Department of Nuclear Engineering, University of California, Berkeley, Berkeley CA, 94720 USA}
\address[lbl]{Lawrence Berkeley National Laboratory,  Berkeley CA, 94720 USA}




\begin{abstract}


\lipsum[1]



\end{abstract}

\begin{keyword}
%% keywords here, in the form: keyword \sep keyword
Update \sep Your \sep Keywords \sep Here



\end{keyword}

\end{frontmatter}

%%
%% Start line numbering here if you want
%%
% \linenumbers

% \listoftodos


%% main text 
\section{Introduction}


\lipsum[1-2]




\section{Experimental methods and materials}


\lipsum[1-2]

\subsection{Subsection 1}


Use citations like this\cite{Voyles2018a}.

\subsection{Subsection 2}

\lipsum[1]





\section{Results}


\lipsum[1-2]



\section{Conclusions}


\lipsum[1-2]

 


 
 \section{Acknowledgements}
 
 List your acknowledgements here...

 


% 
%     
% \pagebreak
% 
% \onecolumn
% 
% %% The Appendices part is started with the command \appendix;
% %% appendix sections are then done as normal sections
% %% \appendix
% 
% %% \section{}
% %% \label{}
% 
% %% References
% %%
% %% Following citation commands can be used in the body text:
% %% Usage of \cite is as follows:
% %%   \cite{key}         ==>>  [#]
% %%   \cite[chap. 2]{key} ==>> [#, chap. 2]
%%

% \twocolumn





%% References with BibTeX database:


% ``Overfull \hbox in .bbl'' message fixed by commenting out ''write.url'' in /usr/share/texlive/texmf-dist/bibtex/bst/elsarticle/elsarticle-num.bst for affected entry types (likely, 'article' and 'book', all but 'phdthesis')


% Note: replace '../../library' with the path to the library.bib file that Mendeley generates for you!

\IfFileExists{../../library.bib}{\bibliography{../../library}}{\bibliography{library}}
% \thispagestyle{fancyTOC}
\bibliographystyle{elsarticle-num}



\end{document}

%%
%% End of file `ari_template.tex'. 