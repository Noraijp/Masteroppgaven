\documentclass[11pt,showpacs,preprintnumbers,footinbib,amsmath,amssymb,aps,prl,groupedaddress,superscriptaddress,showkeys]{revtex4-1}
\usepackage{graphicx}
\usepackage{dcolumn}
\usepackage{bm}
\usepackage[colorlinks=true,urlcolor=blue,citecolor=blue,linkcolor=black]{hyperref}
\usepackage{color}
\usepackage{listings}
\bibliographystyle{apacite}
\usepackage{float}
\bibliographystyle{plain}
\usepackage{mathtools} %
\usepackage{amsmath} %

%\usepackage{listings}
\usepackage{color}
\usepackage{hyperref}
\definecolor{dkgreen}{rgb}{0,0.6,0}
\definecolor{gray}{rgb}{0.5,0.5,0.5}
\definecolor{mauve}{rgb}{0.58,0,0.82}


\lstset{frame=tb,
  language=Java,
  aboveskip=3mm,
  belowskip=3mm,
  showstringspaces=false,
  columns=flexible,
  basicstyle={\small\ttfamily},
  numbers=none,
  numberstyle=\tiny\color{black},
  keywordstyle=\color{blue},
  commentstyle=\color{dkgreen},
  stringstyle=\color{mauve},
  breaklines=true,
  breakatwhitespace=true,
  tabsize=3
}


\begin{document}


\title{\Large{Rapport\\}}


\author{Nora Irene Jensen Pettersen}

\begin{abstract}
\centering
ABSTRACT
\end{abstract}

\maketitle

\newpage

\section{Experimental chapter}

\subsection{Lawrence Berkeley National Laboratory}
\noindent
Being one of the first cyclotrons used to produce medical isotopes, the Berkeley lab is still used to produce radionuclides that are a part of the reache in the medical industry. Lawrence national laboratory is located in the hills, directly over the university of California at Berkeley overlooking San Francisco bay.\\
\\
The 88-Inch cyclotron is used in research on varies of fields including astrophysics and nuclear structure. The accelerator is a 300-ton, K = 140 sector-focused cyclotron with the ability to run with both heavy- and light-ions. The facility of the cyclotron is shown in fig (??), it mainly has five experimental caves where cave 0 is the one used for neutron beams and isotope production[the 88-inch cyclotron]. The cyclotron can accelerate ions up to 70??? MeV 
\subsection{Experiment}
\noindent
The experiment was performed on August 2018 in Lawrence Berkeley National Laboratory where we wanted to study the $^{nat}Zn(n,p)^{64,67}Cu$ reaction. $^{nat}Zn$ was placed in cave 0 (See figure ???) in the 88-Inch Cyclotron.\\
\\
As a beam, we used deuterium with the energy of 16 MeV and 33 MeV which hit a 6 mm thick Beryllium disk with the intention to produce neutron flux by deuterium break up. This can be done on any material, but natural beryllium is a dense medium that has the quality as a good conductor of heat, it is stable and it is impossible to produce a radioactive activation product using deuterium. Therefore, i is the perfect candidate for the break up reaction.\\ There is no excited states in deuterium [$https://www.nndc.bnl.gov/ensdf/EnsdfDispatcherServlet$] and if we give it enough energy, it will break up into a neutron and a proton. \texttt{ Be har en liten bindindseegenskap til nøytroner? 1.5 MeV eller noe? Dette gjør at nøytronene gå ut. hva med protonene?} By tuning the deuterium energy, we can adjust how many neutrons we want to make.\\
\\
This neutron beam was then used to irradiate our target of Zinc and th co-targets Indium, Aluminum, Zirconium, Yttrium and NaCl. Med denne metoden så får du en fokusert nøytron beam The targets were placed 10 cm away from the Beryllium since we only wanted the neutrons to hit the targets without any angular distribution. \\
\\
The co-targets Indium, Aluminum, Zirconium and Yttrium was there as monitor foils since they have known cross sections, which makes it easier to find the absolute cross sections for $^{64,67}Cu$.\\
\\
By irradiating Zinc targets we were able to measure energy-integrated production rate for Cu64 and Cu67, witch is two medically interesting radionuclides. Cu64 has a half life of 12.701 hours and decayes with beta+/EC, which makes is suitable for imaging by PET. 67Cu has a halflife of 68.3 hours and decays with beta-. Since it has a half life of approximately 2.5 days and quite a short range of e- from beta- decay, which irriadiate cancerous cells and spares healthy tissue \\
\\
\subsection{Target design}
\noindent
A specific design of the targets were used to make sure that the cross section for each target could be measured for the different energies we radiated the targets with. Five different targets, Al, In, Y, Zn, and Zr, were made into small “packings” using kapton tape to tightly seal them, bruker dette også til å hinde contamination siden targetset kan være sjør og unngå spredning av løst radiaktivt stoff. The kapton contains polymer carbon, hydrogen and oxygen but the sticky part of the tape are made from silicone. For each target we measured their wight, thickness and length Using what? to calculate the uncertainty of each target. The targets were attached to a plastic frame and were placed in the end of a metallic box made of? where we wanted the targets to be 10 cm from the Br disk, the targets were held in place by a spring between them and the disk in front. The metallic box was then placed in the beamline by inserting it into a vacuumous tube. The vacuum was there to prevent the beam to hit air molecules and to make the beamline as straight as possible, such that more of the beam would be able to hit the targets. The radiation of the targets were on for approximately 12 hours? for both 16 MeV and 33 MeV. \\
\\
nøytroner stopper ikke i materie. stopper nøytroner med høy tetthet pga kollisjoner, høy energi inelastic scattering (> 2 MeV) og for lav energi elastic scattering. Lav Z fordi pga kollisjoner, du må bruke en kjerne som er like store.\\
\\
\subsection{Measurements of gamma rays}
\noindent
The gamma rays was observed using a high purity germanium detector (HPGe). The detector is a semiconductor and therefore it is doped? with? and as a result the germanium detector will be negative charged on one side and positive charged on the other side. do I need this?
When a sample is in the detector, the germanium detector will detect the electrons from the sample. Three things that will occur (if enough energy) is photoelectric effect, compton effect and pair production explain what they are?. In additional to that, there will be peaks that represents different nuclei?\\
\\
We radiated the targets overnight from 23? pm to 08? am with both the 16 MeV beam and the 33 MeV beam


\subsection{Calibration}
\noindent
To do the calibration and everything I used gf3. 





\newpage



\end{document}